\chapter{Kurzfassung}

\begin{german}
	Die Microservice Architektur is ein aufstrebendes Pattern in der Softwareentwicklung.
	Eine Applikation welche mit der Microservice Architektur aufgebaut ist, besteht aus vielen kleineren Services, die genau einen Zweck erfüllen.
	Deswegen werden Funktionsaufrufe innerhalb der Applikation zu Netzwerkaufrufen zwischen den Services.
	Um die Netzwerkkommunikation vor Spoofing zu schützen muss gegenseitige Authentifizierung gewährleistet werden.
	Hierfür werden Service-zu-Service Authentifizierungsmechanismen verwendet.

	Die verbreitetsten Service-zu-Service Authentifizierungsmechanismen sind self-signed JSON Web Tokens (JWT) und mutual TLS (mTLS).
	Diese Arbeit beschreibt die Ideen und Konzepte, sowie die Motivationen und Herausforderungen dieser Mechanismen.
	Außerdem wird ein Projekt vorgestellt, welches beide Mechanismen implementiert und Einblicke über die Implementierung dieses Projekts werden gegeben.
	Mit dem Wissen, aus dieser Arbeit sollen Entwickler in der Lage sein selbst den richtige Authentifizierungsmechanismus für ihre Projekte zu bestimmen.

	Die Vergleiche haben gezeigt, dass beide Mechanismen sehr effizient sind und das selbe Maß an Sicherheit liefern.
	Deswegen ist kein Mechanismus dem anderem gegenüber in jedem Fall überlegen.
	Self-signed JWTs sind der bevorzugte Authentifizierungsmechanismus, falls Nonrepudiation eine Anforderung ist, wenn die Applikation dazu neigt, den User-Context zu benötigen, oder falls die Entwickler die Authentifizierung selbst mit zusätzlichen Parametern adaptieren möchten.
	Falls keine dieser Anforderungen zutrifft, ist mTLS der bevorzugte Authentifizierungsmechanismus, da damit das System so einfach wie möglich gehalten wird.
\end{german}
