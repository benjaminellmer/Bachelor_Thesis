\chapter{Kurzfassung}

\begin{german}
	Die Microservice Architektur ist ein aufstrebendes Pattern in der Softwareentwicklung.
	Eine Applikation welche anhand der Microservice Architektur aufgebaut ist, besteht aus vielen kleinen Services, die genau einen Zweck erfüllen, anstatt aus einer riesigen Komponente.
	Die verschiedenen Services müssen miteinander kommunizieren, um die Logik der anderen Services zu nutzen.
	Somit werden Funktionsaufrufe innerhalb der Applikation zu Funktionsaufrufen über das Netzwerk.
	Diese Netzwerkkommunikation muss Vertraulichkeit, Integrität und Authentisierung gewährleisten.
	Durch die Verwendung des Transport Layer Security (TLS) Protokolls werden Integrität und Vertraulichkeit gewährleistet.
	Außerdem wird anhand des TLS Protokolls der Server gegenüber dem Client authentisiert, jedoch der Client nicht gegenüber dem Server.
	Hierfür werden zusätzliche Authentisierungsmechanismen benötigt.

	Die verbreitetsten Service-zu-Service Authentisierungsmechanismen sind self-signed JSON Web Tokens (JWT) und mutual TLS (mTLS).
	Mutual TLS ist eine Adaptierung des TLS Protokolls und ermöglicht eine effiziente und einfache Implmentierung von Service-zu-Service Authentisierung.
	Andererseits hat man bei mTLS nur wenig Adaptierungsmöglichkeiten, somit ist es schwer für andere Zwecke anpassbar.
	Self-signed JWTs ermöglichen es zusätzliche Parameter in die JWTs zu integrieren.
	Somit kann der Authentisierungsmechanismus so angepasst werden, dass weitere Aufgaben wie das Weitergeben des End-User Contexts vereinfacht werden.
	Zusätzlich haben self-signed JWTs gegenüber mTLS den Vorteil, dass nonrepudiation (Nichtabstreitbarkeit) gewährleistet werden kann, indem die erhaltenen Requests und JWTs gespeichert werden.

	Schlussendlich kann man nicht sagen, dass einer der beiden Authentisierungsmechanismen in jedem Fall dem anderem gegenüber überlegen ist.
	Self-signed JWTs sind der bevorzugte Authentisierungsmechanismus, wenn nonrepudiation eine Anforderung ist, oder wenn die Applikation dazu neigt den End-User Context zu benötigen.
	mTLS ist der bevorzugte Authentisierungsmechanismus, wenn das Ziel ist, so einfach wie möglich Service-zu-Service Authentisierung zu implementieren.
	Trotzdem man nicht sagen kann, dass man mTLS nicht für komplexe Anwendungsfälle geeignet ist, oder dass self-signed JWTs nicht für einfachere Anwendungsfälle geeignet sind.
\end{german}
