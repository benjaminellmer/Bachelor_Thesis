\chapter{Kurzfassung}

\begin{german}
	Die Microservice Architektur ist ein aufstrebendes Pattern in der Softwareentwicklung.
	Eine Applikation welche anhand der Microservice Architektur aufgebaut ist, besteht aus vielen sogenannten Services.
	Jeder Service eines Deployments erfüllt genau einen Zweck.
	Das bedeutet die Applikation besteht aus vielen Services mit weniger Logik anstatt aus einer riesigen Komponente, welche die gesamte Logik implementiert.
	Die verschiedenen Services müssen miteinander kommunizieren um die Logik der anderen Services zu nutzen.
	Somit werden Funktionsaufrufe innerhalb der Applikation zu Funktionsaufrufen über das Netzwerk.
	Diese Netzwerkkommunikation muss Vertraulichkeit, Integrität und Authentisierung gewährleisten.
	Durch die Verwendung des Transport Layer Security (TLS) Protokolls werden Integrität und Vertraulichkeit gewährleistet.
	Ausserdem wird mittels dem TLS Protokoll der Server gegenüber dem Client authentisiert, jedoch der Client nicht gegenüber dem Server.
	Hierfür werden zusätzliche Mechanismen benötigt.

	Die verbreitetsten Service-zu-Service Authentisierungsmechanismen sind self-signed JSON Web Tokens (JWT) und mutual TLS (mTLS).
	Mutual TLS ist eine sehr einfache und effeziente Erweiterung des TLS Protokolls.
	Andererseits hat man bei mTLS nur wenig Adaptierungsmöglichkeiten, somit ist es schwer für andere Zwecke anpassbar.
	Self-signed JWTs ermöglichen es zusätzliche Parameter in die JWTs zu integrieren.
	Somit kann der Authentisierungsmechanismus so angepasst werden, dass weitere Aufgaben wie das weitergeben des End-User Contexts vereinfacht werden.
	Zusätzlich haben self-signed JWTs gegenüber mTLS den Vorteil, dass nonrepudiation gewährleistet werden kann, indem die erhaltenen Requests und JWTs gespeichert werden.

	Schlussendlich kann man nicht sagen, dass einer der beiden Authentisierungsmechanismen in jedem Fall dem anderem gegenüber überlegen ist.
	Self-signed JWTs sind der bevorzugte Authentisierungsmechanismus, wenn nonrepudiation eine Anforderung ist, oder wenn die Applikation dazu neigt den End-User Context zu benötigen.
	mTLS ist der bevorzugte Authentisierungsmechanismus, wenn das Ziel ist, so einfach wie Möglich Service-zu-Service Authentisierung zu implementieren.
	Wobei man nicht sagen kann, dass man mTLS nicht für komplexe Anwendungsfälle verwenden soll und dass man self-signed JWTs nicht für einfache Anwendungsfälle verwenden soll.
\end{german}
