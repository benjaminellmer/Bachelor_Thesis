\chapter{Abstract}
% Thema, Problematik, Motivation
The microservice architecture is a currently emerging pattern in software engineering.
Instead of having one huge application, the logic is split into numerous smaller units that fulfill one single purpose.
Therefore function calls within the application migrate to remote calls over the network.
The remote calls between the services have to provide mutual authentication, so secure the system from intruders.
Therefore service-to-service authentication mechanisms are necessary to provide mutual authentication among the services.

% Lösungsansatz, Methodik
The most popular service-to-service authentication mechanisms are self-signed JSON Web Tokens (JWT) and mutual TLS (mTLS).
This thesis describes the concepts and fundamentals and discusses the motivations and challenges of both mechanisms.
Furthermore a project, which implements the compared authentication mechanisms was reviewed and discussed.
This thesis aims to help developers to choose the correct authentication mechanism for their project.

% Wichtigste Ergebnisse, Erkenntnisse 
Self-signed JWTs are the preferred authentication mechanism, when nonrepudiation is a requirement, when the application tends to share the user-context, or when the developers require to adapt the authentication mechanism with additional parameters.
When none of this requirements apply, mTLS is the preferred approach, since it keeps the system efficient and simple.

