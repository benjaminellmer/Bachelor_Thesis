\chapter{Introduction}
\label{cha:Introduction}
In the past years, a trend towards highly-scalable software systems like the microservice architecture emerged.
The migration from a monolithic architecture towards microservices has enormous consequences regarding the security of software systems~\cite{shmeleva2020microservices}. 
Function calls within the same project migrate to remote calls over the network~\cite{chandramouli2019microservices}. 
%TODO: Unclear antecedent
This offers a larger attack surface because intruders could spoof the communication among the services.
Therefore the communication among the services has to provide mutual authentication to prevent attackers from exploiting the system.
Additionally, service-to-service communication has to provide confidentiality.
Confidentiality is usually addressed using TLS, which also provides authentication, but it only authenticates the server to the client.
Therefore additional authentication mechanisms are necessary to implement mutual authentication.
The most popular approaches for service-to-service authentication are mutual TLS (mTLS) and authentication using self-signed JSON Web Tokens (JWT)~\cite{dias2020microservices}.
This thesis will describe and compare those authentication mechanisms to discover differences, advantages, and disadvantages.

\section{Motivation}
The International Data Corporation (IDC) has predicted that by 2022, 90\% of all apps will feature microservice architectures~\cite{idcprediction2019}. 
So it is inevitable to deal with the numerous mechanisms to secure such systems properly. 
Authentication is one of the most crucial security challenges.
When authentication is neglected, attackers could perform attacks like the Main-in-the-middle-Attack to exploit the system, even if other security challenges like confidentiality and integrity are mastered.
Such attacks could result in substantial data leaks or allow attackers to misuse the system for their advantage.

The microservice architecture is based on having multiple services running in multiple locations.
%TODO: Unclear antecedent
This results in a bigger attack surface because multiple machines are exposed to the internet, making it simpler to find vulnerabilities.
%TODO: Unclear antecedent
This is one of the reasons why Netflix received massive attacks on their microservice based-systems in the past years~\cite{pereira2019security}.

These motivations show how vital microservice security is, and this thesis aims to help microservice developers choose the correct authentication mechanisms for their projects.

\section{Challenges}
Since the microservice architecture became as popular as in the last years, there is a lack of evaluation research and only limited insight into the particular security concerns, especially regarding service-to-service authentication. 
The most popular solutions and existing implementations are closed source, like the mTLS Architecture of Netflix.
Other freely available approaches are often poorly documented and therefore hard to understand~\cite{yarygina2018overcoming}.

Additionally, the migration from the monolithic architecture to the microservice architecture results in a performance overhead~\cite{ueda2016workload}.
Furthermore, the authentication mechanisms produce latencies because additional operations have to be performed.
Therefore it is crucial to choose the correct authentication mechanism and implement it as efficiently as possible.

The microservice architecture and the service-to-service authentication bring some more challenges and difficulties, which will be declared and discussed in the following chapters.

\section{Chapter Overview}
Chapter~\ref{cha:Microservice_Architecture} introduces essential fundamentals of the microservice architecture. 
It should explain why the later discussed authentication mechanisms are a requirement caused by the microservice architecture.

\noindent Chapter~\ref{cha:Related_Work} summarizes the state-of-the-art concepts regarding microservice security and public key-based authentication.
Furthermore, it describes the fundamentals of the required technologies for the later discussed authentication mechanisms.

\noindent Chapter~\ref{cha:authentication_mechanisms} describes the fundamentals and concepts of the compared authentication mechanisms in detail.
Especially the motivations, challenges, and differences between them are analyzed.

\noindent Chapter~\ref{cha:project_structure} shows the backend of an app, which implements the discussed authentication mechanisms.
Additionally, it provides some implementation details for the ASP.Net framework.

\noindent Chapter~\ref{cha:experiment} describes the setup of a performance experiment comparing the discussed authentication mechanisms.
The experimental results are then interpreted and discussed.
