\chapter{Technologies}
This chapter describes the technologies and tools, which are necessary for the implementation of the later discussed authentication mechanisms.


%\section{OpenSSL}
%OpenSSL is an open-source library that provides implementations of the state-of-the-art cryptographic algorithms, including the TLS protocols.
%It is fully cross-platform and can either be used within programs or by its CLI.
%It is crucial to use libraries like OpenSSL for cryptographic operations because most developers are not fully aware of all dangers, resulting in vulnerabilities caused by faulty implementations~\cite{viega2002network}.
%OpenSSL will be used to create and sign keys and certificates in chapter~\ref{cha:Implementation}.

%\subsection{OpenSSL file formats}
%Files with the following extensions are created and used in chapter~\ref{cha:Implementation}.
%The file extensions are not exclusive to the OpenSSL library, and these are only the extensions that are later used.
%There are many more available extensions, which are not discussed in this thesis~\cite{opensslextensions, viega2002network}:
%\begin{description}
%    \item[pem:] A Privacy Enhanced Mail file is a base64 encoded text file, which can be used as a container for a private key, or one to many certificates.
%    \item[crt:] A pem file containing certificates can use the crt extension instead of the pem extension.
%    \item[key:] A pem file containing a private key can use the key extension instead of the pem extension.
%    \item[csr:] A csr file contains a certificate signing request intended to be signed by a certificate authority.
%    \item[pfx:] A pfx file is a container that can include certificates and a private key together in a binary file.
%\end{description}
