\chapter{Related Work}
\label{cha:Related_Work}
This chapter summarizes the state-of-the-art concepts for microservice security, public key-based authentication, and the later required technologies.
Service-to-service authentication is only one part of microservice security.
Therefore all related topics, considering microservice security, are discussed.
Furthermore, the concepts of public-key authentication are explained since the authentication mechanisms compared in this thesis are based on public-key cryptography.

\section{Microservice Security}
Siriwardena and Dias~\cite{dias2020microservices} gave an extensive guide of all topics related to microservice security. 
They separated the security of a microservice deployment into edge-level security and service-level security.
Since the microservice architecture splits the backend into multiple minor services, it is insufficient to secure the system only on the edge-level or only on the service-level.
Each part of the system must be appropriately secured regarding authentication, authorization, confidentiality, and integrity.

\subsection{Edge-level security}
Edge-level security is defined as the security mechanisms that protect the resources within the deployment from attackers located outside the deployment. 
The API Gateway is responsible for edge security. 
Therefore, it is the only entry point to the microservice deployment.
It intercepts all requests targeted for the APIs of the services.
After validating the requests, it dispatches the valid ones to the microservices.
The main tasks of an API Gateway are authentication of the end-user, authorization, and throttling~\cite{dias2020microservices}.
It authenticates the end-user using access tokens coming from access delegation technologies like OAuth 2.0 or Open ID Connect~\cite{siriwardena2014advanced}.
By outsourcing the end-user authentication to the API Gateway, the user has to authenticate himself only once and not multiple times to each service~\cite{dias2020microservices}.
The API Gateway could handle authorization for all requests, but in most cases, it is performed on both levels, the edge-level, and the service level. 
Therefore the API Gateway is only responsible for coarse authorization assertions, preventing a single-point-of-decision~\cite{barabanov2020authentication}.

\subsection{Service-level security}
Service-level security is defined as the security mechanisms that protect the communication among the microservices.
According to Barabanov and Makrushin~\cite{barabanov2020authentication}, service-level security can be decomposed into the sub-functions service-level authentication, service-level authorization, and external identity propagation.
Service-level security can either be implemented by the microservices themselves or by a service-mesh.
A service mesh can be seen as a dedicated infrastructure layer that manages the service-to-service communication of containerized services.
In a typical microservice deployment with a service mesh, each microservice has its service proxy that works in a transparent manner~\cite{dias2020microservices}.
The service mesh takes care of service discovery, routing, load balancing, traffic configuration, authentication, authorization, and monitoring~\cite{chandramouli2019microservices}.
Therefore the services can focus on exactly the tasks they are intended for and do not have to care about security-related tasks~\cite{dias2020microservices}.

\subsubsection{Service-to-service authentication} 
\label{sec:service-to-service-authentication}
Authentication is the process of identifying the communication partner to protect a system from spoofing.
Since the microservices communicate using remote calls, their communication has to provide mutual authentication~\cite{dias2020microservices}.
Service-to-service authentication can be implemented in the following ways~\cite{dias2020microservices}:
\begin{itemize}
    \item Trust the Network (TTN)
    \item Mutual Transport Layer Security (mTLS)
    \item Self-signed JSON Web Tokens (JWTs)
\end{itemize}
Trust the Network is a security approach based on the assertion that nobody has access to the components within a network perimeter.
All components rely on network security, but internal misbehavior could lead to exploits allowing attackers to intrude into the network perimeter and exploit the microservices~\cite{zaheer2019eztrust}. 
Therefore the industry is heading towards zero-trust networks, and the TTN approach is not more used as primary authentication mechanism~\cite{barabanov2020authentication}.

Service-to-service authentication based on mTLS and self-signed JWTs will be discussed in more detail in chapter~\ref{cha:authentication_mechanisms}.

\subsubsection{Service-level authorization} 
\label{sec:service-level-authorization}
Authorization defines the tasks that a principal is allowed to perform on a system.
It requires that the principal is already authenticated because the authorization is performed based on the identity~\cite{siriwardena2014advanced}. 
Service-level authorization gives the microservices more control to enforce access control.
Authorization is usually performed using policy decision point (PDP) models like the centralized PDP model or the embedded PDP model~\cite{dias2020microservices, barabanov2020authentication}.
Proper service-to-service authentication mechanisms are a precondition for service-to-service authorization since the authorization can be bypassed with insufficient authentication~\cite{siriwardena2014advanced}.

\subsubsection{External entity identity propagation} 
\label{sec:external-entity-identity-propagation}
In order to perform the authorization correctly, the services have to know the context of the caller.
The most popular technic for identity propagation is extracting the user's context within JSON Web Tokens.
The tokens are passed between the microservices and the API Gateway.
The propagated identity of the user can be extracted from the token, and the token's signature must be checked.
The microservices can then perform authorization based on the identity of the client~\cite{barabanov2020authentication, dias2020microservices}.
The way how identity propagation is implemented depends on the authentication mechanism used.
%unclear antecedent
This will be discussed in more detail in chapter~\ref{cha:authentication_mechanisms}.

\section{Public key-Based Authentication}
Authentication can be achieved in multiple ways.
Two common approaches are symmetric cryptography and public-key cryptography.
Both authentication mechanisms discussed in chapter~\ref{cha:authentication_mechanisms} are based on public-key cryptography.
Public key cryptography provides higher security than symmetric cryptography.
Therefore it is the preferred method to implement authentication mechanisms, although it requires higher computation and communication costs than symmetric cryptography~\cite{pubkeycrypto}.

\subsection{Public key cryptography}
Public key cryptography is also called asymmetric cryptography because the main idea is that different keys are used for encryption and decryption~\cite{anderson2020security}.
Each participant is required to own at least one key pair.
A key pair consists of a public key available to everyone and a private key that is only known by the owner.
Encrypting a message with the public key allows many people to encrypt messages in a way that only the person who owns the private key can read them.
Furthermore, it allows one person to encrypt messages in a way that many people can read them by encrypting the message with the private key~\cite{henriques2017using}.
% Unclear antecedent
This is also known under the term digital signature. 
Digital signatures can be used to provide authentication and integrity for messages~\cite{anderson2020security}.

The commonly used algorithm for digital signatures is RSA.
RSA is based on factoring, its encryption key consists of the modulus $N$, which is hard to factor.
The modulus $N$ is calculated by multiplying two large prime numbers $p$ and $q$ with each other.
Additionally the encryption key has a public factor $e$ that has no common factors with either $p-1$ or $q-1$.
The private key consists of the factors $p$ and $q$, that have to be kept secret~\cite{anderson2020security}.
The person that knows the private key can encrypt a message using the following formula, where $M$ is the message, and $C$ is the encrypted message~\cite{anderson2020security}:
\begin{displaymath}
	C = M^e (mod N)
\end{displaymath}
An encrypted message can be decryped using the following formular:
\begin{displaymath}
	M = \sqrt[e]{C (mod N)}
\end{displaymath}
Only the owner of the private key can simply calculate the message from the cipher, using the Fermat's little theorem\footnote{See~\cite{fermatlittle} for further information}.

The problem with asymmetric cryptography is that the encryption and decryption times are worse than using symmetric cryptography.
The restriction to use only asymmetric cryptography could decrease the system's performance.
Therefore it is common sense to use hybrid systems as done in TLS.
In TLS, public-key cryptography is used for authentication, but symmetric cryptography is used to provide confidentiality for the communication~\cite{henriques2017using}.
Henriques~\cite{henriques2017using} showed that the performance of systems could be improved using a hybrid approach.

\subsection{Public Key Infrastructure} \label{sec:pki}
Public key cryptography assumes that the receiver of a message already knows and trusts the sender's public key.
Public Key Infrastructures (PKI) are used to achieve this assertion.
They are responsible for providing a possibility to retrieve the public key in a trusted way.
% Unclear antecedent
This is done using Certificate Authorities (CA) and certificates.
CAs sign certificates, and each communication partner who trusts the CA trusts the certificates signed by it.
For this purpose, usually, X.509 certificates are used~\cite{anderson2020security}.

A PKI can either be an open PKI (global) or a closed PKI (self-hosted).
Closed PKIs have a specific bounded context~\cite{hlavaty2003risk}.
A common use case for closed PKIs are microservice deployments because they are usually company intern and have a specific bounded context~\cite{dias2020microservices}.
The project reviewed in chapter~\ref{cha:project_structure} makes use of a closed PKI created with OpenSSH.
Closed PKIs are a popular option because they allow risk management and provide secrecy of its code.
Open PKIs can be inspected by the public, and based on the inspection, it is determined whether the PKI is trusted or not.
Certificates are retrieved by making partnerships with CAs.
The main advantage is that no proprietary software is needed since the PKIs are managed by the PKI vendors~\cite{hlavaty2003risk}.

\subsection{Key Management} \label{sec:key_management}
Key management is a requirement for public key based authentication. 
It results in being the most challenging part for the later discussed mechanisms.
When the key management is fragile, it will have consequences for all parts of the system.
Especially service-to-service authentication is affected by the quality of the key management.
The authentication mechanism can not stay secure when the keys of the participants are compromised~\cite{dias2020microservices, fumy1993principles}.
According to Fumy et al.~\cite{fumy1993principles}, a key management service has to implement the following tasks:
\begin{description}
	\item[Entity Registration:] The service must provide a procedure to create a link between an authenticated identity and its keys.
	\item[Key Generation:] The service must provide a procedure to create key pairs with good cryptographic quality.
	\item[Certification:] The service must provide a procedure for issuing certificates. Certification is often a part of key distribution.
	\item[Authentication/Verification:] The service must provide a procedure to guarantee entity authentication, message content authentication, and message origin authentication.
	\item[Key Distribution:] The service must provide a procedure to supply keys for parties legitimately asking for them.
\end{description}

Dias and Siriwardena~\cite{dias2020microservices} furthermore give insights into the key provisioning (distribution) process of Netflix.
Netflix uses its own broker called Lemur for the key provisioning.
It is performed in the following steps, visualized in figure~\ref{fig:key_provisioning_netflix}:
\begin{enumerate}
    \item During the continuous delivery process, each microservice gets a set of credentials that are good enough to access the Lemur APIs.
% Unclear antecedent
		This is done using a Netflix internal tool called Metatron.
		Metatron credentials are long-lived credentials. 
		They can be used for a longer time period, and the following steps can be repeated multiple times.
    \item The microservice talks to the Lemur API to obtain a signed certificate for its credentials.
% Unclear antecedent
		This can happen either during the startup process of the microservice or when the microservice is rotating its keys.
    \item Lemur creates a Certificate Signing Request (CSR) addressed to the CA.
    \item The certificate is signed using the CA.
		Lemur is not a CA, but it knows how to integrate with a CA to generate signed certificates.
    \item Lemur returns the signed certificate to the microservice, who can then use it to authenticate itself to other services.
\end{enumerate}
Therefore the developers do not have to worry about creating and signing certificates.
Instead, they have to implement the communication with the Lemur API.


\begin{figure}
	\centering
	\begin{sequencediagram}
		\newthread{A}{:Microservice}{}
		\newinst[2]{B}{:Metatron}{}
		\newinst[2]{C}{:Lemur}{}
		\newinst[2]{D}{:CA}{}

		\begin{call}{A}{obtain credentials}{B}{credentials}
		\end{call}
		\begin{call}{A}{obtain certificate}{C}{signed certificate}
			\begin{call}{C}{CSR}{D}{signed certificate}
			\end{call}
		\end{call}
	\end{sequencediagram}
	\caption{Key provisioning of netflix using Lemur and Metatron~\cite{dias2020microservices}}
	\label{fig:key_provisioning_netflix}
\end{figure}

%\begin{figure}
%	\centering
%	\includegraphics{images/related-work/netflix-provisioning.pdf}
%	\caption{Key provisioning of netflix using Lemur and Metatron~\cite{dias2020microservices}}
%	\label{fig:key_provisioning_netflix}
%\end{figure}

\section{Technologies}
\subsection{X509.Certificate}
X.509 certifcates bind the subject of a certificate to a public key.
They are used to assure the user of a certificate that the certificate's subject owns the corresponding private key.
The most significant advantage of certificates is that they can be exchanged using untrusted communication channels because the signature becomes invalid when the content of a certificate is changed.
Therefore manipulations can be detected, and manipulated certificates can be declined~\cite{x509rfc}.

When a client wants to consume a service hosted on a server, it must obtain the server's certificate.
If the client does not know the public key of the CA who signed the server's certificate, he has to obtain it.
Obtaining the public key often results in chains because the client may have to work his way up until he reaches a CA he trusts.
Such chains are also called certification paths~\cite{x509rfc}.

Depending on the version of the standard, a certificate can include more or less information.
The information is always stored inside the tbsCertificate, signatureAlgorithm, and signatureValue fields and can be expanded using extensions~\cite{x509rfc}.
\begin{description}
	\item[tbsCertificate:] contains the data of the certificate, including the subject, the issuer, the public key of the subject, the validity period, and additional information~\cite{x509rfc}.
	\item[signatureAlogrithm:] declares the cryptographic algorithm used to sign the certificate.
		Algorithms are identified by their unique "OBJECT IDENTIFIER".
		The most commonly used algorithms are the RSA algorithm and the Digital Signature Algorithm (DSA)~\cite{x509rfc}.
	\item[signatureValue:] contains the value of the digital signature.
		It is obtained by signing the content of the tbsCertificate, using the algorithm specified in the signatureAlgorithm field.
		The signature is used to verify the validity of the information embedded in the tbsCertificate field~\cite{x509rfc}.
\end{description}

\subsubsection{Certificate Revocation}
Certificate revocation is one of the most significant downsides of certificates.
It results in requiring to communicate with a centralized authority for each request~\cite{dias2020microservices}.
A certificate can be revocated for the following reasons~\cite{dias2020microservices}: 
\begin{itemize}
    \item The Private key of the CA is compromised
    \item The Private key of the microservice is compromised
    \item The holder of the certificate is no longer the identity who requested the certificate 
    \item The CA finds out that the parameters provided in the CSR are invalid
\end{itemize}
The hardest part about certificate revocation is informing all participants about the revocation of a certificate.
% Unclear antecedent
This is done using certificate revocation lists (CRL) or other revocation mechanisms.
The big downside of CRLs is that each CA has to store a list of revocated certificates and the clients have to retrieve this list whenever they establish a connection to a server.
This causes high latencies that can be reduced by caching.
Otherwise, caching also reduces security because a certificate can be revocated during the lifetime of the cache.
Especially when the CA does not respond to the CRL query is tough to handle~\cite{dias2020microservices}.


\subsection{JSON Web Token}
A JSON Web Token (JWT) is a container that can carry authentication assertions, authorization assertions, and further information in a cryptographically safe manner.
An authentication assertion can be anything that identifies the user.
Usually, usernames or e-mail addresses are used as authentication assertions.
An authorization assertion can be any information about the access permissions.
For example, a JWT can include the information, whether the user is an admin or an unprivileged user~\cite{dias2020microservices}. 

\subsubsection{Structure}
A JWT is decomposed into the header, the payload, and the signature, that are concatenated and separated by a dot~\cite{jwtdocauth0}.
%A valid JWT could look like the JWT shown in figure~\ref{fig:myjwt}.
%\begin{figure}
%    \textcolor{red}{Header}.
%	\textcolor{blue}{Payload}.
%	\textcolor{darkgreen}{Signature} \\ \\
%    \textcolor{red}{eyJhbGciOiJIUzI1NiIsInR5cCI6IkpXVCJ9}.
%	\textcolor{blue}{eyJzdWIiOiIxMjM0NTY3ODkiLCJpYXQi\\OjE1MTYyMzkwMjIsInVzZXJuYW1lIjoiYmVuamFtaW4uZWxsbWVyIiwiZW1haWw\\iOiJiZW5qYW1pbi5lbGxtZXJAeWFob28uY29tIiwiYWRtaW4iOmZhbHNlfQ}.
%	\textcolor{darkgreen}{0ksqN7\\1oloNvq3IrY7w72uoTgPz9Gpn08p-KSbFulY0}
%    \caption{Sample JSON Web Token}
%    \label{fig:myjwt}
%\end{figure}

The \textbf{header} contains metadata related to the JWT. 
The metadata is usually the type of the token and the signature algorithm.
The specification defines that only HS256\footnote{HMAC SHA-256} and the none algorithm must be implemented by conforming JWT implementation.
It is recommended to additionally implement the algorithms RS256 and ES256\footnote{Elliptic Curve Digital Signature Algorithm (ECDSA) with 256-bit key}~\cite{jwtdocauth0, jwtrfc}.
The base64 encoded header is the first part of the JWT.

The \textbf{payload} is a set of registered and custom claims.
A claim is a piece of information about an entity.
The JWT specification defines registered claims that are not mandatory for all cases but should provide a good starting point for a set of valuable claims to ensure interoperability.
The software architects can define custom claims on their own, depending on their needs.
The custom claims registered in the IANA registry are called public claims, and those not registered in the IANA registry are called private claims~\cite{jwtdocauth0, jwtrfc}.
The base64 encoded payload is the second part of the JWT.

The chosen signature algorithm signs the base64 encoded header, the base64 encoded payload, and a secret (only with symmetric encryption like HMAC).
The \textbf{signature} provides integrity for the message, and if it was signed with a private key, it additionally provides authentication~\cite{jwtdocauth0}.
This mechanisms is later used for service-to-service authentication using self-signed JWTs.
The base64 encoded signature is the third part of the JWT.

\subsection{Transport Layer Security}
The Transport Layer Security (TLS) Protocol provides authentication, integrity, and confidentiality for the communication between two parties.
It consists of two layers, the handshake protocol and the record protocol~\cite{turnertls}.

\subsubsection{Handshake Protocol}
The handshake protocol is responsible for negotiating a cipher suite and providing authentication using X.509 certificates.
The cipher suite declares the key exchange algorithm, the signature algorithm, the symmetric encryption algorithm, including the mode of the encryption algorithm and the hashing algorithm~\cite{turnertls, kurbatov2021design}.
The handshake varies on the key exchange method, but it can be separated into the following steps~\cite{krawczyk2013security}:
\begin{enumerate}
    \item The server and the client exchange Hello messages.
    \item The server sends its certificate to the client.
    \item The client sends a pre-master secret to the server.
    \item The client and the server finish the handshake, using the independently computed master secret.
\end{enumerate}

\subsubsection{Record Protocol}
The record protocol provides a secure channel for the communication between two parties.
% Unclear antecedent
This is done by using the algorithms declared in the cipher suite.
Confidentiality is assured using symmetric encryption.
Integrity is provided using Message Authentication Codes (MAC)~\cite{kurbatov2021design, krawczyk2013security}.

\subsubsection{mTLS} \label{sec:mtls}
TLS itself is also called one-way TLS because it allows the client to identify the server but not the server to identify the client.
Therefore mTLS was introduced to provide authentication in both directions.
The server and the client must own a private/public key pair.
Therefore it is more appropriate for the communication between systems instead of the communication between users and servers~\cite{dias2020microservices}. 
When mTLS is used, the client must present his certificate to the server during the TLS handshake.
The detailed steps are discussed in more detail in chapter~\ref{cha:authentication_mechanisms}.

\section{Conclusion}
This chapter described the fundamentals to understand the details and differences between the later described authentication mechanisms.
Furthermore, it was described why service-to-service authentication is needed and what additional mechanisms have to be implemented to secure a microservice deployment.

It is essential to understand that the whole system is compromised when one security mechanism does not work properly.
For example, authorization can be affected when authentication is neglected.
Each service and each layer of the deployment has to be appropriately secured.
The problem with the microservice architecture is that one compromised service can be used to compromise other services.
Therefore, it is crucial to choose the correct security mechanisms and understand how they work. 
