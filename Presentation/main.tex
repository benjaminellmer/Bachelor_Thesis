% DefaultFontSize: 10pt for conferences, 8/9pt for lectures
\def \DefaultFontSize{9pt}

% PresTextLayout (hyphenation, justification)
%   Options:
%      0   Don't modify anything, set text as-is (no hyphenation/justification)
%      1   Never ever hyphenate anything, but justify text.
%      2   Hyphenate only when absolutely necessary, and justify text.
%      3   Hyphenate words whenever possible and justify text.
\def \PresTextLayout{2}


% Set path to template and include it
\def \templatepath{FH-Beamer-Template}
\input{\templatepath/fhooe-beamer-wide}

% Data for first slide - should be self-explanatory
\def \PresTitle      {Service-to-service authentication in a microservice deployment}
\def \PresShortTitle {Bachelor Thesis Presentation}
\def \PresSubTitle   {FH Hagenberg, WS 2021/2022}
\def \PresAuthor     {Benjamin Ellmer}
\def \PresAuthorTitle{Benjamin Ellmer}
\def \PresHeading    {Bachelor Thesis Presentation}
\def \PresFooter     {Mobile Computing}

% Cover image, position, height and caption
\def \IntroImage     {FH-Beamer-Template/fh-logo-wide.jpg}
\def \IntroImageXPos {33}
\def \IntroImageHeight{2cm}
\def \IntroImageText {}

\newfontscheme

% That's it, let's go for some slides
\begin{document}

% Auto-generate cover slide using info provided above
\fhfirstslide

% Section title = Top title
\section{Microservice Architecture}

\subsection{Monolithic vs. Microservice Architecture}
\fhslide{
	\fhdualpage{.5}{.5}{
		This text will be on the left side of the dual-page, and takes up
		40 \% of the pagewidth. Lorem ipsum dolor
		sit amet, consectetur adipisici elit, ...
	}{
		This text will be on the right side of the dual-page, and takes up
		60 \% of the page width. Lorem ipsum dolor
		sit amet, consectetur adipisici elit, sed eiusmod tempor incidunt ut
		labore et dolore magna aliqua.
	}
}

% Sub-Section title = Sub-Top title
\subsection{Presenting a regular slide}
\fhslide{
	This is a regular slide

	\fhblock{Block title}{
		And a basic block for emphasizing important information, using
		\cmd{\textbackslash{}fhblock\{Title\}\{Text\}}. Below is
		an example of the \cmd{\textbackslash{}fhdualpage} command, which
		lets you define the pane proportions and content for each pane.
	}

	\vspace{5mm}
	% Page proportions 4:6
	\fhdualpage{.4}{.6}{
		This text will be on the left side of the dual-page, and takes up
		40 \% of the pagewidth. Lorem ipsum dolor
		sit amet, consectetur adipisici elit, ...
	}{
		This text will be on the right side of the dual-page, and takes up
		60 \% of the page width. Lorem ipsum dolor
		sit amet, consectetur adipisici elit, sed eiusmod tempor incidunt ut
		labore et dolore magna aliqua.
	}
}

\subsection{Presenting a list-slide with \textbackslash{}fhslist}
\fhslist{
	\item Many slides actually contain only multiple bullets, rather than
	text in paragraphs
	\item In order to keep it simple, theres the \cmd{\textbackslash{}fhslist}
	command, which begins a new slide and opens an itemize environment
	for your convenience.
	\item After starting the slide with the command above, you can immediately
	continue to provide \cmd{\textbackslash{}item} bullets.
	\fhlist{
		\item For sub-bullets, there's no need for begin/end blocks either.
		Simply use \cmd{\textbackslash{}fhlist} command and drop your
		\cmd{\textbackslash{}item}s.
		\item By the way, monospaced and grey-backgrounded text can be
		done with \cmd{\textbackslash{}cmd\{code-like text\}}
	}

	\ctr{Centered text via \cmd{\textbackslash{}ctr\{\}}}

	\fhref{Bottom-references fia \cmd{\textbackslash{}fhref\{\}}}
}

\subsection{Topic slides}
\fhtopicslide{\textbackslash{}fhtopicslide\{content\} renders a new
slide, e.g. for a new topic, with large bold font and vertically centered
text.}

\subsection{Images, essential slides, and other macros}
\fhslist{
	\item There are a lot more easy-to-use macros
	\item Just look at the first 40-something lines of the template tex
	source file
	\item There are commands for URL references, including images,
	quad-pages, aligned math equations, fragile slides (e.g. for lstlistings),
	etc.

	\item Easy image inclusion with \cmd{\textbackslash{}fhimg[width]\{path\}}
	\fhcimg[4cm]{\templatepath/android-attack2.png}

	\item See the burning-hot icon on the top left? You can mark slides as
	important using the \cmd{\textbackslash{}essential} command.

	\essential

	\item Theres also a \cmd{\textbackslash{}moot} command, if you want to tag
	supplementary slides. Of course, the images can be easily changed - just
	change the filepath in the source, or overwrite the image itself in the
	template folder.
}

\subsection{That's it - have fun}
\fhslist{
	\item Just look at the provided \cmd{example.tex} file, how all these
	commands are used in-source -- it's really easy
	\item For a more complete list of available commands and macros, look
	at the template source-code
	\item If you find errors, have suggestions or even patches for new
	commands and macros you think others can benefit from, don't hesitate
	to tell me: \cmd{erik.sonnleitner@fh-hagenberg.at}
}

\end{document}
