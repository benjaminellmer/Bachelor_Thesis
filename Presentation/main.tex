% DefaultFontSize: 10pt for conferences, 8/9pt for lectures
\def \DefaultFontSize{9pt}

% PresTextLayout (hyphenation, justification)
%   Options:
%      0   Don't modify anything, set text as-is (no hyphenation/justification)
%      1   Never ever hyphenate anything, but justify text.
%      2   Hyphenate only when absolutely necessary, and justify text.
%      3   Hyphenate words whenever possible and justify text.
\def \PresTextLayout{2}


% Set path to template and include it
\def \templatepath{FH-Beamer-Template}
\input{\templatepath/fhooe-beamer-wide}

% Data for first slide - should be self-explanatory
\def \PresTitle      {Service-to-service authentication in a microservice deployment}
\def \PresShortTitle {Bachelor Thesis Presentation}
\def \PresSubTitle   {FH Hagenberg, WS 2021/2022}
\def \PresAuthor     {Benjamin Ellmer}
\def \PresAuthorTitle{Benjamin Ellmer}
\def \PresHeading    {Bachelor Thesis Presentation}
\def \PresFooter     {Mobile Computing}

% Cover image, position, height and caption
\def \IntroImage     {FH-Beamer-Template/fh-logo-wide.jpg}
\def \IntroImageXPos {33}
\def \IntroImageHeight{2cm}
\def \IntroImageText {}

\newfontscheme

% That's it, let's go for some slides
\begin{document}

% Auto-generate cover slide using info provided above
\fhfirstslide

% Section title = Top title
\section{Microservice Architecture}

\subsection{Swapindo}
\fhslide{
	\fhdualpage{.6}{.4}{
		\fhlist{
			\item Flea market app developed in android (so far)
			\item Easy and playful swapping
			\item Simple and safe lending and renting
			\item Fast and secure buying and selling
			\item All-in-one-platform as mobile application
			\item Backend based on the microservice architecture
			}
	} {
		\fhcimg[4cm]{images/logo.png}
	}

	\fhcimg[6.5cm]{images/triple.png}

}

\subsection{Monolithic Architecture vs. Microservice Architecture}
\fhslide{
	\fhdualpage{.3}{.7}{
		\fhcimg[3cm]{images/TikZ_Monolith.pdf}
	}{
		\fhcimg[9cm]{images/TikZ_Microservice.pdf}
	}
}

%\subsection{Monolithic Architecture}
%\fhslist{
%	\item All functions are managed and served in one place
%	\item Large codebase and a lack of modularity
%	\item Easy development, debugging and testing
%	\item Become very complex, when they scale up
%	\item Making changes becomes very hard in such large and complex applications
%	\item When new technologies are applied, most parts of the application have to be rewritten
%}


\subsection{Microservice Architecture}
\fhslist{
	\item All approaches that could centralize business logic are avoided
	\item Declaration of service boundaries is very hard
	\item Implementation details are hidden, to avoid coupling between services
	\item Failures can be isolated
	\item Language level calls become remote calls
	\vspace{5mm}
	\fhdualpage{.4}{.6}{
		\centering
		Language-Level Call:
		\fhcimg[5cm]{images/call_01.png}
	}{
		\centering
		Remote Call:
		\fhcimg[8cm]{images/call_02.png}
	}
}


\subsection{Remote Calls and Authentication}
\fhslide{
	\fhblock{Term Description}{
		\textbf{Confidentiality} guarantees that only the sender and the receiver are allowed to read and understand the communication between them. \\
		\textbf{Integrity} guarantees the trustworthiness and correctness of data, by providing the ability to detect unauthorized modifications. \\
		\textbf{Authentication} is the process of identifying the communication partner to protect a system from spoofing. 
	}
	\fhlist{
		\item Remote calls have to provide confidentiality, integrity and authentication
		\item Confidentiality and integrity are assured using TLS
		\item TLS also authenticates the server to the client, but not the client to the server 
		\item Most popular service-to-service authentication mechanisms are mutual TLS and self-signed JWTs
		\item Trust the network is a deprecated approach for service-to-service authentication
	}
}

\subsection{Mutual TLS}
\fhslide{
	\fhlist{
		\item Each service owns a certificate, signed by a Certificate Authority 
		\item The certificates are exchanged during the TLS-handshake
		\item When the signature of the certificate is valid, the service is trusted
		\item Efficient and straightforward
		\item Based on the TLS protocol, therefore hard to adapt for other purposes like sharing the user-context
	}
	\fhdualpage{.3}{.4}{
		\fhcimg[4cm]{images/TikZ_mTLS_base_structure.pdf}
	}{
		\fhcimg[6cm]{images/tls-handshake.png}
	}
}

\subsection{Self-signed JWT}
\fhslide{
	\fhlist{
		\item Each service owns a key pair
		\item A JWT signed with the private key has to be embedded within the Authorization header
		\item Additional parameters can be embedded within the JWT
		\item User-context is usually shared using a nested JWT
		\item Nonrepudiation can be achieved by storing the received JWTs and requests
		\fhdualpage{.2}{.4}{
			\fhcimg[4cm]{images/TikZ_jwt_base_structure.pdf}
		}{
			\fhcimg[6cm]{images/jwt.png}
		}
	}
}

\subsection{Comparison Result}
\fhslide{
	\textbf{mTLS:}
	\fhlist{
		\item Each service owns a key pair
	}
	\vspace{5mm}
	\textbf{Self-signed JWT:}
	\fhlist{
		\item Each service owns a key pair
	}
}


\subsection{Best authentication mechanism for your project}
\fhslide{
	\fhlist{
		\item Is the microservice architecture fitted for your project?
		\item Is nonrepudiation an requirement ? $\rightarrow$ \textbf{self-signed JWT}
		\item Does your application tend to share the user-context? $\rightarrow$ \textbf{self-signed JWT}
		\item Does your authentication require custom parameters ? $\rightarrow$ \textbf{self-signed JWT}
		\item None of the previous reasons ? $\rightarrow$ \textbf{mTLS}
	}
}

% Sub-Section title = Sub-Top title
\subsection{Presenting a regular slide}
\fhslide{
	This is a regular slide

	\fhblock{Block title}{
		And a basic block for emphasizing important information, using
		\cmd{\textbackslash{}fhblock\{Title\}\{Text\}}. Below is
		an example of the \cmd{\textbackslash{}fhdualpage} command, which
		lets you define the pane proportions and content for each pane.
	}

	\vspace{5mm}
	% Page proportions 4:6
	\fhdualpage{.4}{.6}{
		This text will be on the left side of the dual-page, and takes up
		40 \% of the pagewidth. Lorem ipsum dolor
		sit amet, consectetur adipisici elit, ...
	}{
		This text will be on the right side of the dual-page, and takes up
		60 \% of the page width. Lorem ipsum dolor
		sit amet, consectetur adipisici elit, sed eiusmod tempor incidunt ut
		labore et dolore magna aliqua.
	}
}

\subsection{Presenting a list-slide with \textbackslash{}fhslist}
\fhslist{
	\item Many slides actually contain only multiple bullets, rather than
	text in paragraphs
	\item In order to keep it simple, theres the \cmd{\textbackslash{}fhslist}
	command, which begins a new slide and opens an itemize environment
	for your convenience.
	\item After starting the slide with the command above, you can immediately
	continue to provide \cmd{\textbackslash{}item} bullets.
	\fhlist{
		\item For sub-bullets, there's no need for begin/end blocks either.
		Simply use \cmd{\textbackslash{}fhlist} command and drop your
		\cmd{\textbackslash{}item}s.
		\item By the way, monospaced and grey-backgrounded text can be
		done with \cmd{\textbackslash{}cmd\{code-like text\}}
	}

	\ctr{Centered text via \cmd{\textbackslash{}ctr\{\}}}

	\fhref{Bottom-references fia \cmd{\textbackslash{}fhref\{\}}}
}

\subsection{Topic slides}
\fhtopicslide{\textbackslash{}fhtopicslide\{content\} renders a new
slide, e.g. for a new topic, with large bold font and vertically centered
text.}

\subsection{Images, essential slides, and other macros}
\fhslist{
	\item There are a lot more easy-to-use macros
	\item Just look at the first 40-something lines of the template tex
	source file
	\item There are commands for URL references, including images,
	quad-pages, aligned math equations, fragile slides (e.g. for lstlistings),
	etc.

	\item Easy image inclusion with \cmd{\textbackslash{}fhimg[width]\{path\}}
	\fhcimg[4cm]{\templatepath/android-attack2.png}

	\item See the burning-hot icon on the top left? You can mark slides as
	important using the \cmd{\textbackslash{}essential} command.

	\essential

	\item Theres also a \cmd{\textbackslash{}moot} command, if you want to tag
	supplementary slides. Of course, the images can be easily changed - just
	change the filepath in the source, or overwrite the image itself in the
	template folder.
}

\subsection{That's it - have fun}
\fhslist{
	\item Just look at the provided \cmd{example.tex} file, how all these
	commands are used in-source -- it's really easy
	\item For a more complete list of available commands and macros, look
	at the template source-code
	\item If you find errors, have suggestions or even patches for new
	commands and macros you think others can benefit from, don't hesitate
	to tell me: \cmd{erik.sonnleitner@fh-hagenberg.at}
}

\end{document}
