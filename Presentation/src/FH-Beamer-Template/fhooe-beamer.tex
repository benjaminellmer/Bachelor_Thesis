% =============================================================================
% FH corporate design LaTeX slideset template (unofficial)
% Licence: GPLv3
%
%   (c) DI Dr. Erik Sonnleitner, 2012 - 2015
%       es@delta-xi.net / erik.sonnleitner@fh-hagenberg.at
%       (legacy: esonnleitner@faw.jku.at)
% University of Applied Sciences Upper Austria, Hagenberg
% Version: 3.0 (2015-09-01)
% =============================================================================
%
% Version history:
%  * 3.0 (2015-09-01)
%     - \fhtopicslide{topic} - blank slide with \huge topic h/v centered 
%     - \CodeAsm: set language to asm
%     - \cmd - \texttt + grey background
%     - \fhfragplain - like \fhfrag, but with black slide (no heading, etc)
%     - \fhfrag  - fragile slide environment (!)
%     - \fhdpt - dualpage, aligned on top
%     - \fheq - equation env
%     - \fhalign - align env
%     - \fhalignx - align* env
%     - \fhmp - minipage
%     - \fhcmp - centered minipage
%     - \essential, \moot markers
%     - \fhquadpage
%  * 2.1 (2015-08-12)
%     - \fhcenter, \fhright, \fhleft
%     - \fhcimg (centered image)
%     - \fhurlref
%     - \ctr
%  * 2.0 (2015-08-04)
%     - Converted to FH
%  * 1.1 (2012-02-07)
%     - altered enumerate labels (strictly arabic numbers)
%     - added \fhblock
%  * 1.0 (2012-01-25)
%     - first public release


\documentclass[\DefaultFontSize,xcolor={dvipsnames}]{beamer}
\usepackage[ngerman,english]{babel}
%\usepackage[english]{babel}
\usepackage{microtype}
\usepackage[utf8]{inputenc}
\usepackage{amsmath,amsfonts,amssymb}
%\usepackage{mathabx}
\usepackage{wrapfig}
\usepackage{color}
%\usepackage{ragged2e}
\usepackage{ifthen}
\usepackage{xifthen}
\usepackage{bbding}
\usepackage{tikz}
\usetikzlibrary{shapes.callouts,decorations.pathmorphing}
\usepackage{xparse}
\usepackage{xcolor}
\usepackage{fixltx2e}
\usepackage{tabularx}
\usepackage{listings}
\usepackage{keystroke}
\usepackage{verbatim}


\usepackage{helvet}

% Main presentation layout definition, done by theme packages
\usetheme{Berlin}
  % AnnAbor, ANtines, Bergen, Berkeley, Berlin, Boadilla, boxes, CambrigsUS,
  % Copenhagen, Darmstadt, default, Dresden, Frankfurt, Goettingen, Hannover,
  % Ilmenau, JuanLesPins, Luebeck, Madris, Malmoe, Marburg, Montepellier,
  % PaloAlto, Pittsburgh, Rochester, Singapore, Szeged, Warsaw

\usecolortheme{seagull}
%\usecolortheme{whale}
%\usecolortheme{orchid}
  % albatross, beaver, beetle, crane, default, dolphin, dove, fly, lily, orchid,
  % rose, seagull, seahorse, sidebartab, structure, whale, wolverine

\usefonttheme{professionalfonts}
  % default, professionalfonts, serif, structurebold, structureitalicserif,
  % structuresmallcapsserif

\useinnertheme{rounded}
  % circles, default, inmargin, rectangles, rounded

\useoutertheme{split}
%\useoutertheme{infolines}
  % default, infolines, miniframes, shadow, sidebar, smoothbars, smoothtree,
  % split, tree

\usepackage{beamerthemeboxes}
% \usepackage{beamerthemeboxes}
  % beamerthemebars, beamerthemeclassic, beamerthemelined, beamerthemeplain,
  % beamerthemeshadow, beamerthemesidebar, beamerthemesidebar-tab, 
  % beamerthemesidebardark, beamerthemesidebardark-tab, beamerthemetree,
  % beamerthemetree-bars, beamerthemesplit, beamerthemeboxes

% define template path
% update 2015/09/08: use \provide rather than \new, so the new command is only
%   defined, if the presentation source does not do so
%\newcommand{\templatepath}{../../../FH-Beamer-Template}
\providecommand{\templatepath}{../../../FH-Beamer-Template}


% The following stuff is for the institute/department logo, at top right corner
	\usepackage[absolute,overlay]{textpos}
	%\only<1>{
	\pgfdeclareimage[height=1.2cm]{logo}{\templatepath/fh-ooe.png}
	\setlength{\TPHorizModule}{43mm}
	\setlength{\TPVertModule}{1mm}
	\newcommand{\MyLogo}

	\newcommand*\oldmacro{}%
	\let\oldmacro\insertshorttitle%
	\renewcommand*\insertshorttitle{%
	  \MyLogo
	  \oldmacro\hfill}


% Remove navigation symbols, add slide numbering instead
	\setbeamertemplate{navigation symbols}{}
	% \setbeamertemplate{footline}[page number]

% Set the footline
	\setbeamertemplate{footline}{
		\leavevmode
		\hbox{
			\begin{beamercolorbox} [
				wd=.7\paperwidth,
				ht=2.5ex,
				dp=1.125ex,
				leftskip=.3cm,
				rightskip=.3cm plus1fill
			] {author in head/foot}

				\usebeamerfont{author in head/foot} \PresAuthor - \emph{ \PresFooter }
			\end{beamercolorbox}

			\begin{beamercolorbox} [
				wd=.3\paperwidth,
				ht=2.5ex,
				dp=1.125ex,
				leftskip=.3cm plus1fil,
				rightskip=.1cm
			]{title in head/foot}

			\usebeamerfont{title in head/foot} \#\insertframenumber/\inserttotalframenumber
		  \end{beamercolorbox}
		}
		\vskip0pt
	}

% Set text margins left and right, force to use much space for text
\setbeamersize{text margin left=.5cm,text margin right=.5cm}

% Add institute/department logo to headline
\addtobeamertemplate{headline}{} { \MyLogo }
%\addtobeamertemplate{headline}{} { }

% Set institute logo
%\logo{\includegraphics[height=0mm]{\templatepath/institute-logo.png}}

% Define new column types for tabularx: fixed-width left/centered/right
\newcolumntype{L}[1]{>{\raggedright\arraybackslash}p{#1}}
\newcolumntype{C}[1]{>{\centering\arraybackslash}p{#1}}
\newcolumntype{R}[1]{>{\raggedleft\arraybackslash}p{#1}}

% Define color for our red hrule and bullets
\definecolor{alizarin}{rgb}{0.76, 0.13, 0.28}

% Define a square for the itemize labels
\newsavebox{\mysquare}
\savebox{\mysquare}{\textcolor{alizarin}{\rule[.5mm]{1mm}{1mm}}}

% Set itemize labels
\setbeamertemplate{itemize item}{\usebox{\mysquare}}
\setbeamertemplate{itemize subitem}{ {\color{alizarin}---} }
\setbeamertemplate{itemize subsubitem}{$\bullet$}

\setbeamertemplate{enumerate item}[default]

% Set itemize/enumerate font sizes
\setbeamerfont{itemize body}{size=\small}
%\setbeamerfont{itemize subbody}{size=\footnotesize}
\setbeamerfont{itemize subbody}{size=\scriptsize}
\setbeamerfont{itemize subsubbody}{size=\scriptsize}

\setbeamerfont{enumerate body}{size=\small}
\setbeamerfont{enumerate subbody}{size=\footnotesize}
\setbeamerfont{enumerate subsubbody}{size=\scriptsize}



% -----------------------------------------------------------------------------
% Hyphenation (based on \PresTextLayout variable)
% -----------------------------------------------------------------------------
% Text layouting (hyphenation, justification)
%   Options:
%      0   Don't modify anything, set text as-is (no hyphenation/justification)
%      1   Never ever hyphenate anything, but justify text.
%      2   Hyphenate only when absolutely necessary, and justify text.
%      3   Hyphenate words whenever possible and justify text.


\ifthenelse{ \PresTextLayout = 0} {
	\hyphenpenalty=10000000
}{}
\ifthenelse{ \PresTextLayout = 1} {
	\usepackage{ragged2e}
	\hyphenpenalty=10000000
	\justifying
}{}
\ifthenelse{ \PresTextLayout = 2} {
	\usepackage{ragged2e}
	\hyphenpenalty=500
	\justifying
}{}
\ifthenelse{ \PresTextLayout = 3} {
	\usepackage{ragged2e}
	\hyphenpenalty=0
	\justifying
}{}

% -----------------------------------------------------------------------------
% FH Macros
% -----------------------------------------------------------------------------


% \fhrule - a red horizontal rule
\newcommand{\fhrule}{
		% {\color{alizarin} \rule{12cm}{1pt} \\ } \vspace{0.2cm}
		\vspace{-0.15cm}
		\normalsize{
			{\color{alizarin} \rule{ 10cm }{1pt} \\ } \vspace{0.2cm}
		}
}

% LEGACY -- use \fhfirstslide now (
% \fhintroslide - the very first slide of the presentation
\newcommand{ \fhintroslide } {
	% First frame, presentation frontpage
	\begin{frame}
		\frametitle{ \small{ \PresShortTitle } }

		\begin{flushright}
			\large{\textbf{ \PresTitle } }

			\vspace{0.2cm}
			% Meta-info, Author
			\normalsize{
				\PresSubTitle \\
				\PresAuthorTitle
			}

			{\color{alizarin}  \rule{7cm}{1pt}\\ }

			\vspace{0.2cm}
			\large{
				\PresHeading
			}

		\end{flushright}
	\end{frame}
}

% XXX TESTESTEST XXX
% \fhfirstslide: Build introduction slide by \def's given in src file
\newcommand{\fhfirstslide}{
	\begin{frame}
		\frametitle{ \small{ \PresShortTitle } }
		\begin{flushright}
			\large{\textbf{ \PresTitle } }

			\vspace{0.2cm}
			% Meta-info, Author
			\normalsize{
				\PresSubTitle \\
				\PresAuthorTitle
			}

			{\color{alizarin}  \rule{7cm}{1pt}\\ }

			\vspace{0.2cm}
			\large{
				\PresHeading
			}

		\end{flushright}

		\begin{textblock}{0}(0.1,{\IntroImageXPos})
		%\begin{textblock}{0}(0.1,30.3)
			\includegraphics[height=\IntroImageHeight]{\IntroImage}\\
				\begin{textblock}{100}(0.1,85)
					\includegraphics[width=1cm]{\templatepath/cc-by-nc-sa.png}\\
					\vspace{-0.15cm}
					{\tiny \IntroImageText}
					%{\tiny "Rogue Android": CC-BY-NC-SA via http://picphotos.net}
				\end{textblock}
		\end{textblock}
	\end{frame}
}


% \ssec: shortcut for \subsection
\newcommand{\ssec}[1]{\subsection{#1}}

% \fhslide environment -- this saves time & money ;)
\newcommand{\fhslide}[2][] {
	\transfade
	%\begin{frame}[shrink]
	%\begin{frame}[allowpagebreak,t]
	\begin{frame}[t]
		% \frametitle{ \subsecname \\ \small{ \secname} }
		\frametitle{ \small{ \secname} }
		\textbf{ \large{ \subsecname } } \\
		%\ifthenelse{ \equal{#1}{\string noline}}
		\ifthenelse{ \isempty{#1} }
		  { \fhrule }
		  {}
		#2
	\end{frame}
}

% \fhtopicslide: New topic, centered, large, textbf
\newcommand{\fhtopicslide}[1]{
	\fhslide{
		\vspace{2.5cm}
		 \ctr{{\Large \hi{#1}}}
	}
}


\newenvironment{fhfslide}[1]
{\begin{frame}[fragile]{}
\frametitle{\small{\secname}}
\textbf{\large{\subsecname}}\newline
\fhrule
#1
}
{\end{frame}}


\newenvironment{fhfrag}
% XXX: old \fhfrag had allowframebreks!
 %{\begin{frame}[t,allowframebreaks=0.8,fragile,environment=fhfrag]\frametitle{\small{\secname}}\textbf{\large{\subsecname}}\newline\fhrule}
 {\begin{frame}[t,fragile,environment=fhfrag]\frametitle{\small{\secname}}\textbf{\large{\subsecname}}\newline\fhrule}
 {\end{frame}}

\newenvironment{fhfragnb}
 {\begin{frame}[t,fragile,environment=fhfragnb]\frametitle{\small{\secname}}\textbf{\large{\subsecname}}\newline\fhrule}
 {\end{frame}}

% fragile with *nothing* else (no headline, ruler, etc)
\newenvironment{fhfragplain}
 {\begin{frame}[t,squeeze,fragile,environment=fhfragplain]}
 {\end{frame}}



%% \fhslidef environment -- "fragile" version
%\newcommand{\fhslidef}[1]{
%\begin{fhfslide}
%#1
%\end{fhfslide}
%}

%% \fhslidef environment -- "fragile" version
%\newcommand{\fhslidef}[2][] {
%	\transfade
%	%\begin{frame}[shrink]
%	%\begin{frame}[allowpagebreak,t]
%	\begin{frame}[t,fragile]
%		% \frametitle{ \subsecname \\ \small{ \secname} }
%		\frametitle{ \small{ \secname} }
%		\textbf{ \large{ \subsecname } } \\
%		%\ifthenelse{ \equal{#1}{\string noline}}
%		\ifthenelse{ \isempty{#1} }
%		  { \fhrule }
%		  {}
%		#2
%	\end{frame}
%}

% \fhlist
\newcommand{ \fhlist }[1] {
	\begin{itemize}
	#1
	\end{itemize}
}

% \fhenum
\newcommand{ \fhenum}[1] {
	\begin{enumerate}
	#1
	\end{enumerate}
}


% \fhdpt - \fhdualpage with text on top
\newcommand{\fhdpt}[4]{
	\begin{columns}[t]
        \column{#1\textwidth}
			#3
        \column{#2\textwidth}
			#4
	\end{columns}
}

% \fhdualpage
\newcommand{ \fhdualpage }[4]{
	\begin{columns}
        \column{#1\textwidth}
			#3
        \column{#2\textwidth}
			#4
	\end{columns}
}

% \fhtriplepage
\newcommand{ \fhtriplepage }[6]{
	\begin{columns}
        \column{#1\textwidth}
			#4
        \column{#2\textwidth}
			#5
        \column{#3\textwidth}
			#6
	\end{columns}
}

% \fhquadpage
\newcommand{ \fhquadpage }[8]{
	\begin{columns}
        \column{#1\textwidth}
			#5
        \column{#2\textwidth}
			#6
        \column{#3\textwidth}
			#7
        \column{#4\textwidth}
			#8
	\end{columns}
}

% \fhblock
\newcommand{ \fhblock}[2] {
	\begin{block}{#1}
	#2
	\end{block}
}

% \fhfig: filename, width, caption, label
\newcommand { \fhfig }[4]{
	\begin{figure}
		\includegraphics[width=#2]{#1}
		\caption{#3}
		\label{#4}
	\end{figure}
}

% \fhimg: like figure, but without caption+label (e.g. for design only)
\newcommand{\fhimg}[2][0.92\paperwidth]{
	\includegraphics[width=#1]{#2}
}

\newcommand{\fhcimg}[2][0.92\paperwidth]{
	\begin{center}
		\fhimg[#1]{#2}
	\end{center}
}

% just damn center
\newcommand{\ctr}[1]{
	\begin{center}#1\end{center}
}


% \fhref: reference
\newcommand { \fhref }[1]{
	\begin{textblock}{1.8}(0.1,90)
		\begin{scriptsize}
			\textbf{References:} #1
		\end{scriptsize}
	\end{textblock}
}

% \fhurlref: url reference
\newcommand { \fhurlref }[1]{
	\begin{textblock}{1.8}(0.1,90)
		\begin{scriptsize}
			\textbf{References:} \url{#1}
		\end{scriptsize}
	\end{textblock}
}


% \essential: mark slide content as essential
\newcommand{\essential}{
	\begin{textblock}{0}(0,1.5)
		\includegraphics[height=.5cm]{\templatepath/important2.png}
	\end{textblock}
}

% \moot: mark slide content as moon/not for learning
\newcommand{\moot}{
	\begin{textblock}{0}(0.01,2.5)
		%\includegraphics[height=.4cm]{../_template/crossed-blur.png}
		\includegraphics[height=.4cm]{\templatepath/crossed-blur.png}
	\end{textblock}
}




% \fhslist: slide with only a list
\newcommand{\fhslist}[1]{
	\fhslide{
		\fhlist{
			#1
		}
	}
}

% \fhcenter
\newcommand{\fhcenter}[1]{
	\begin{center}
		#1
	\end{center}
}

% \fhright
\newcommand{\fhright}[1]{
	\begin{flushright}
		#1
	\end{flushright}
}

% \fhleft
\newcommand{\fhleft}[1]{
	\begin{flushleft}
		#1
	\end{flushleft}
}

% \fhmp - minipage
\newcommand{\fhmp}[2]{
	\begin{minipage}{#1}
		#2
	\end{minipage}
}

% \fhcmp - centered minipage
\newcommand{\fhcmp}[2]{
	\begin{center}
		\fhmp{#1}{#2}
	\end{center}
}

% \fhalign
\newcommand{\fhalign}[1]{
	\begin{align}
		#1
	\end{align}
}

% \fhalignx
\newcommand{\fhalignx}[1]{
	\begin{align*}
		#1
	\end{align*}
}

% \fheq
\newcommand{\fheq}[1]{
	\begin{equation}
		#1
	\end{equation}
}


% \cmd: alias for \texttt
%\newcommand{\cmd}[1]{\texttt{#1}}

% \cmd (new): Texttt + grey background
\newcommand\cmd[2][]{~\tikz[overlay]\node[fill=gray!20,inner sep=2pt, anchor=text, rectangle, rounded corners=1mm,font=\ttfamily,#1]{#2};\phantom{\texttt{#2}}\hspace{3pt}}



\definecolor{shadecolor}{rgb}{1,0.8,0.3}
% White box for covering stuff
\newcommand{\whitebox}[2]{
	\begin{textblock}{0}(0.0,#1)
	{\color{white}\rule{2.5cm}{#2}}
	\end{textblock}
}

\definecolor{mygreen}{rgb}{0,0.6,0}
\definecolor{mygray}{rgb}{0.5,0.5,0.5}
\definecolor{mymauve}{rgb}{0.58,0,0.82}



%-Comment bubbles--------------------------------------------------------------
\pgfkeys{%
    /calloutquote/.cd,
    width/.code                   =  {\def\calloutquotewidth{#1}},
    position/.code                =  {\def\calloutquotepos{#1}}, 
    author/.code                  =  {\def\calloutquoteauthor{#1}},
    /calloutquote/.unknown/.code  =  {\let\searchname=\pgfkeyscurrentname
                                 \pgfkeysalso{\searchname/.try=#1,
    /tikz/\searchname/.retry=#1},\pgfkeysalso{\searchname/.try=#1,
                                  /pgf/\searchname/.retry=#1}}
}

\newcommand\calloutquote[2][]{%
       \pgfkeys{/calloutquote/.cd,
         width               = 5cm,
         position            = {(0,-1)},
         author              = {}}
  \pgfqkeys{/calloutquote}{#1}
  \node [rectangle callout,callout relative pointer={\calloutquotepos},
         text width=\calloutquotewidth,/calloutquote/.cd, #1]
        (tmpcall) at (0,0) {#2};
  \node at (tmpcall.pointer){\calloutquoteauthor};
}

\newcommand\bash[2][]{\tikz[overlay]\node[fill=gray!20,inner sep=2pt, anchor=text, rectangle, rounded corners=1mm,font=\ttfamily,#1]{#2};\phantom{\texttt{#2}}\hspace{3pt}}

% Basic orientation for bubble arrows
\newcommand{\TopLeft}{(-1,1)}
\newcommand{\TopRight}{(1,1)}
\newcommand{\BottomLeft}{(-1,-1)}
\newcommand{\BottomRight}{(1,-1)}


% Usage: \bubble{position (x,y)}{bubble width}{arrow direction}{text}
\newcommand{\bubble}[4]{
	\begin{textblock}{0}#1
		\begin{tikzpicture}
			\calloutquote[width=#2,position={#3},fill=LimeGreen!90,rounded corners]{#4}
		\end{tikzpicture} 
	\end{textblock}
}
%-End Bubble Code--------------------------------------------------------------


\newcommand{\hi}[1]{\textbf{#1}}


% -- lst listings --
\definecolor{keywords}{HTML}{8A4A0B}
\definecolor{background}{HTML}{EEEEEE}
\definecolor{comments}{HTML}{868686}
\definecolor{myblue}{RGB}{20,105,176}

% language definition for assembly (rudimentary)
\lstdefinelanguage{asmx86}{
	keywords={.section,.data,.text, .globl, movl, je, cmpl, jle, jmp,
		int, incl},
	comment=[l]{\#},
	alsoletter={.}, % add . as char to be used in keywords
}

\lstset{
	%basicstyle=\small\ttfamily,
	basicstyle=\footnotesize\ttfamily,
	backgroundcolor=\color{background},
	numbers=left,
	firstnumber=0,
	stepnumber=2,
	numberstyle=\tiny\color{mygray},
	numbersep=4pt,
	commentstyle=\color{comments},
	frame=lines,
	frameround=ftff,
	tabsize=4,
	keywordstyle=\color{myblue}\textbf,
	language=asmx86
}

\newcommand{\CodeAsm}{
	\lstset{
		language=asmx86
	}
}


