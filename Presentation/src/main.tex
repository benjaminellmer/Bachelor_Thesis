% DefaultFontSize: 10pt for conferences, 8/9pt for lectures
\def \DefaultFontSize{10pt}

% PresTextLayout (hyphenation, justification)
%   Options:
%      0   Don't modify anything, set text as-is (no hyphenation/justification)
%      1   Never ever hyphenate anything, but justify text.
%      2   Hyphenate only when absolutely necessary, and justify text.
%      3   Hyphenate words whenever possible and justify text.
\def \PresTextLayout{2}


% Set path to template and include it
\def \templatepath{FH-Beamer-Template}
\input{\templatepath/fhooe-beamer-wide}

% Data for first slide - should be self-explanatory
\def \PresTitle      {Service-to-service authentication in a microservice deployment}
\def \PresShortTitle {Bachelor Thesis Presentation}
\def \PresSubTitle   {FH Hagenberg, WS 2021/2022}
\def \PresAuthor     {Benjamin Ellmer}
\def \PresAuthorTitle{Benjamin Ellmer}
\def \PresHeading    {Bachelor Thesis Presentation}
\def \PresFooter     {Mobile Computing}

% Cover image, position, height and caption
\def \IntroImage     {FH-Beamer-Template/fh-logo-wide.jpg}
\def \IntroImageXPos {33}
\def \IntroImageHeight{2cm}
\def \IntroImageText {}

\newfontscheme

% That's it, let's go for some slides
\begin{document}

% Auto-generate cover slide using info provided above
\fhfirstslide

% Section title = Top title

\section{Introduction} % 10 Sekunden

\subsection{Swapindo} % 1 Minute
\fhslide{
	\fhdualpage{.5}{.5}{
		\fhlist{
			\item Flea market app developed for android (so far)
			\item Easy and playful swapping
			\item Simple and safe lending and renting
			\item Fast and secure buying and selling
			\item Backend based on the microservice architecture developed in C\# with ASP.NET Core
		}
	} {
		\fhcimg[4cm]{images/logo.png}
		\vspace{2mm}
		\fhcimg[6cm]{images/triple.png}
	}
}

\section{Microservice Architecture}
\subsection{Service-to-service authentication} % 2 Minuten
\fhslide{
	\fhblock{Term Description}{
		\textbf{Authentication} is the process of identifying the communication partner to protect a system from spoofing. 
	}
	\fhdualpage{.55}{.45}{
		\fhlist{
			\item Function calls become remote calls
			\item Remote calls have to provide authentication
			\item TLS authenticates the server to the client, but not the client to the server 
			\item Most popular mechanisms are mutual TLS and self-signed JWTs
		}
	}{
		\fhimg[6.5cm]{images/TikZ_Microservice.pdf}
	}
}

%\subsection{Microservice Architecture}
%\fhslide{
%	\fhdualpage{.55}{.45}{
%		\fhlist{
%			\item All approaches that could centralize business logic are avoided
%			\item Declaration of service boundaries is very hard
%			\item Implementation details are hidden, to avoid coupling between services
%			\item Failures can be isolated
%			\item Language level calls become remote calls
%
%		}
%	}{
%		\fhimg[6.5cm]{images/TikZ_Microservice.pdf}
%	}
%	\vspace{3mm}
%	\fhdualpage{.4}{.6}{
%		\centering
%		Language-Level Call:
%		\fhcimg[5cm]{images/call_01.png}
%	}{
%		\centering
%		Remote Call:
%		\fhcimg[8cm]{images/call_02.png}
%	}
%}



%\subsection{Remote Calls and Authentication}
%\fhslide{
%	\fhblock{Term Description}{
%		\textbf{Authentication} is the process of identifying the communication partner to protect a system from spoofing. 
%	}
%	\fhlist{
%		\item Remote calls have to provide authentication
%		\item TLS authenticates the server to the client, but not the client to the server 
%		\item Most popular service-to-service authentication mechanisms are mutual TLS and self-signed JWTs
%		\item Trust the network is a deprecated approach for service-to-service authentication
%	}
%}

\section{Authentication Mechanisms}

\subsection{Mutual TLS} % 2 Min
\fhslide{
	\fhlist{
		\item Each service owns a certificate, signed by a Certificate Authority 
		\item The certificates are exchanged during the TLS-handshake
		\item When the signature of the certificate is valid, the service is trusted
		\item Efficient and straightforward
		\item Based on the TLS protocol, therefore hard to adapt for other purposes like sharing the user-context
	}
	\fhdualpage{.3}{.4}{
		\fhcimg[4cm]{images/TikZ_mTLS_base_structure.pdf}
	}{
		\fhcimg[6cm]{images/tls-handshake.png}
	}
}

\subsection{Self-signed JWT} % 2 Min - 2:30 Min
\fhslide{
	\fhlist{
		\item Each service owns a key pair
		\item A JWT signed with the private key has to be embedded within the Authorization header
		\item Additional parameters can be embedded within the JWT
		\item User-context is usually shared using a nested JWT
		\item Nonrepudiation can be achieved by storing the received JWTs and requests
		\fhdualpage{.2}{.4}{
			\fhcimg[4cm]{images/TikZ_jwt_base_structure.pdf}
		}{
			\fhcimg[6cm]{images/jwt.png}
		}
	}
}

%\subsection{Performance Experiment} % grob 1:30 wahrscheinlich eher 2 min oder 2:30
%\fhslide{
%
%	\fhdualpage{.5}{.5}{
%		\fhcimg[8cm]{images/experiment-results.png}
%	}{
%		\fhcimg[6cm]{images/experiment-trend.png}
%	}
%}

\subsection{Key Management} % ca. 1:30
\fhslist{
	\item Most challenging part of both mechanisms
	\item Consequence of public key cryptography
	\item Certification
	\item Generate/Distribute keys
	\item Rotate keys
	\item Simple in the beginning, gets hard as the number of services grows
}

\subsection{Choosing the correct mechanism} % 2 Min
\fhslide{
	\fhblock{Hint}{Before thinking about the correct authentication mechanism, the developers should evaluate, if the microservice architecture is the correct technology for the project.}

	\fhlist{
		\item Is nonrepudiation an requirement ? $\rightarrow$ \textbf{self-signed JWT}
		\item Does your application tend to share the user-context? $\rightarrow$ \textbf{self-signed JWT}
		\item Does your authentication mechanism require custom adaptations ? $\rightarrow$ \textbf{self-signed JWT}
		\item Does the project require to work without TLS? $\rightarrow$ \textbf{self-signed JWT}
		\item None of the previous reasons ? $\rightarrow$ \textbf{mTLS}
	}
}


\subsection{End}
\fhtopicslide{Thank you for your attention!}

\end{document}
