\chapter{Introduction}
\label{cha:Introduction}
The trend towards highly-scalable software systems, like the microservice architecture emerged in the past years.
The migration from the monolithic approach towards microservices has enormous consequences regarding the security of software systems~\cite{shmeleva2020microservices}. 
Language-Level calls, which are calls within the same project migrate to remote calls over the network~\cite{chandramouli2019microservices}. 
This offers a larger attack surface because intruders could spoof the communication among the services.
Therefore authentication and confidentiality between services have to be implemented to secure the service-to-service communication.
The confidentiality is usually provided using TLS, but the authentication requires the use of additional authentication mechanisms.
The authentication is usually implemented using mutual TLS~\cite{dias2020microservices}, but the authentication can also be provided using JSON Web Tokens.
Therefore this thesis will describe and compare this authentication mechanisms to work out their differences, advantages and disadvantages.
%Even if microservices deployments are hidden behind an API Gateway, the zero-trust network approach is currently state-of-the art~\cite{dias2020microservices}. 
%This means even if nobody from outside the deployment should be able to communicate with the services, the services do not trust any request.

\section{Motivation}
The International Data Corporation (IDC) has predicted that by 2022, 90\% of all apps will feature microservice architectures~\cite{idcprediction2019}. 
So it is inevitable to deal with the numerous mechanisms to secure such systems properly. 
Authentication is one of the most important security challenges.
Even if confidentiality is provided, when authentication is neglected intruders could perform attacks like the Man-in-the-middle-Attack to exploit the system.
Such attacks could result in huge data leaks or could even allow attackers to abuse the system for their advantages.

The microservice architecture is based on having multiple services running on multiple servers.
This results in a bigger attack surface, because multiple machines are exposed to the internet making it simpler to find vulnerabilities.
This is one of the reasons why Netflix received massive attacks on their microservice based-systems in the past years~\cite{pereira2019security}.

This shows how vital microservice security is and the aim of this thesis is helping the microservice developers to choose the correct authentication mechanisms for their projects.

\section{Challenges}
Since the microservice architecture became as popular as in the last years, there is a lack of evaluation research and only limited insight into the particular security concerns, especially regarding service-to-service authentication. 
The most popular solutions and existing implementations are closed source, like the mTLS\footnote{mutual Transport Layer Security is described in~\ref{sec:mtls}} Architecture of Netflix.
Other freely available approaches are often poorly documented and therefore hard to understand~\cite{yarygina2018overcoming}.
The migration from the monolithic architecture to the microservice architecture results in a performance overhead~\cite{ueda2016workload}.
Furthermore, the security measures produce high latencies because the services have to communicate to centralized authorities.
So the mechanisms have to be implemented very efficiently, and caching should to be applied to minimize the resulting performance overhead.
The microservice architecture and the service-to-service authentication bring some more challenges and difficulties, which will be declared and discussed in the following chapters.

\section{Chapter Overview}
Chapter~\ref{cha:Microservice_Architecture} introduces important fundamentals of the microservice architecture. 
It should give an idea why the later discussed authentication mechanisms are a requirement caused by the microservice architecture.

Chapter~\ref{cha:Related_Work} summarizes state of the art concepts regarding microservice security and public key based authentication.
Furthermore it describes the fundamentals of the required technologies for the later discussed authentication mechanisms.

Chapter~\ref{cha:authentication_mechanisms} describes the two compared authentication mechanisms in detail.
Especially the advantages and disadvantages between those authentication concepts are discussed.

Chapter~\ref{cha:project_structure} shows the backend of an app, which implements the discussed authentication mechanisms.
Additionally it provides some implementation details using the ASP.Net framework.

Chapter~\ref{cha:experiment} describes the setup of a performance experiment of the authentication mechanisms.
The experimental results are then interpreted and discussed.
