\chapter{Final Remarks}
\label{cha:final_remarks}

\section{Discussion}
This thesis explained and compared two authentication mechanisms for the service-to-service security in a microservice deployment.
Both mechanisms are based on public key cryptography, but each mechanism has its own advantages and disadvantages.

Mutual TLS is a very efficient and straightforward authentication mechanism.
The implementation of mTLS is very simple, since the work is handled by the TLS protocol.
mTLS does not provide many configuration parameters and it does not allow to add custom functionalities, like sharing the user context without another technology.
Nevertheless if an developer aims to implement only service-to-service authentication, mTLS is the superior authentication mechanism.

As soon as nonrepudiation is an requirement self signd JWTs are the superior authentication mechanism.
Furthermore JWTs make the identity propagation more convenient and allow the developers to customize the authentication mechanism and add additional parameters.
On the other hand the implementation of self-signed JWTs is more challenging and requires each service to know how to work with JWTs.
Therefore choosing JWTs over mTLS would be unnecessary overhead when the target is implementing only service-to-service authentication.
The decision if the additional control of the approach using self-signed JWTs is worth the overhead has to be evaluated for each project independently.

The biggest challenge of both authentication mechanisms is the key-management.
Both mechanisms require a PKI and require to handle all associated key management tasks.
Therefore the implementation of the authentication mechanisms is much less work than the key management.
Nevertheless, the level of security that is provided using public key cryptography, is worth the expenses.

\section{Summary}
Service-to-service authentication is a requirement caused by the migration to the microservice architecture.
The function calls within the monolithic backend migrate to remote calls.
The remote calls have to assure authentication, confidentiality and integrity.
Confidentiality and integrity can be assured using TLS, but authentication of both parties requires additional mechanisms.

This thesis compared two of the most pupular authentication mechanisms for service-to-service authentication.
Therefore the fundamentals of both mechanisms were discussed and the concepts of the two authentication mechanisms were described in detail.
Additionally an project that implements the authentication mechanisms was reviewed and a experiment that should give an assumption about the performance overhead of the authentication mechanisms was performed and interpreted.

