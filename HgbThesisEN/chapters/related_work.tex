\chapter{Related Work}
\label{cha:Related_Work}
This chapter summarizes the state-of-the-art concepts for microservice security and for authentication in the digital world.
Furthermore the technologies, which are used to implemented this concepts are described in more detail.

\section{Microserivce Security}
Siriwardena and Dias~\cite{dias2020microservices} gave an exhensive guide of all topics related to microservice security. 
They separated the security of an microservice deployment into edge-level security and service-level security.
The microservice architecture is based on separating the system in multiple parts.
It is not sufficient to secure the system only on the edge-level or only on the service-level.
Each part of the system has to be secured properly regarding confidentiality, authentication and authorization.

\subsection{Edge-level security}
Edge-level security is defined as the security mechanisms that protect the resources within the deployment from attackers outside the deployment.
The API Gateway is responsible for the edge security it is the only entry point to the microservice deployment.
All requests targeted for the APIs of the services are intercepted by the API Gateway.
After validation of the requests, it dispatches the requests to the microservices.
The main tasks of an API Gateway are authentication of the end user, authorization, and throttling.
It authenticates the end user using access tokens, which come from access delegation technologies like OAuth 2.0 or Open ID Connect~\cite{siriwardena2014advanced}.
Due to outourcing the end user authentication to the API Gateway, the end user authentication has to be performed only once and not by every service.
The API Gateway is able to perform simple authorization assertions. 
As soon as the authorization gets more granular, the microservices have to perform it on their own, because the API Gateway does not know the business logic.

\subsection{Service-level security}
Service-level security is defined as the security mechanisms that protect the communication among the microservices.
According to Barabanov and Makrushin~\cite{barabanov2020authentication} service-level security can be decomposed into the sub functions service-level authentication, service-level authorization, and external identity propagation.
Service-level security can either be implemented by the microservices themselves or by a service-mesh.
A service mesh can be seen as a dedicated infrastructure layer, which manages the service-to-service communication of containerized services.
In a typical microservice deployment with a service mesh, each microservice has its service proxy, which works transparently~\cite{dias2020microservices}
The servce mesh is responsible for service discovery, routing, load balancing, traffic configuration, authentication authorization and monitoring~\cite{chandramouli2019microservices}.

\subsubsection{Service-to-service-authentication} 
\label{sec:service-to-service-authentication}
Authentication is the process of identifying the communication partner to protect a system from spoofing.
Since the communication among microservices is done using remote calls, their communication has to provide authentication.
Service-to-service authentication can be implemented in the following ways~\cite{dias2020microservices}:
\begin{itemize}
    \item Trust the Network (TTN)
    \item mutual Transport Layer Security (mTLS)
    \item self signed JSON Web Tokens (JWTs)
\end{itemize}
Trust the Network is a security approach, that is based on the assertion that nobody has access to the components within a network perimeter.
All components rely on the network security.
Nevertheless, internal misbehaviour can lead to exploits allowing attackers intrude into the network perimeter and exploit the microservices~\cite{zaheer2019eztrust}. 
Therefore the industry is heading towards zero-trust networks and the TTN approach became deprecated~\cite{dias2020microservices}.

Service-to-servie authentication based on mTLS and self signed JWTs will be discussed in more detail in chapter~\ref{cha:authentication_mechanisms}.

\subsubsection{Service-level authorization} 
\label{sec:service-level-authorization}
Authorization defines the tasks that a principal is allowed to perform on a system.
It requires that the principal is already authenticated because the authorization is performed based on the identity~\cite{siriwardena2014advanced}. 
Service-level authorization gives the microservices more control to enforce access control.
The authorization is usually performed using policy decision point (PDP) models like the centralized PDP model or the embedded PDP model~\cite{dias2020microservices, barabanov2020authentication}.
Proper service-to-service mechanisms are a precondition for service-to-service authorization, since the authorization can be bypassed with insufficient authentication~\cite{siriwardena2014advanced}.

\subsubsection{External entity identity propagation} 
\label{sec:external-entity-identity-propagation}
In order to perform the authorization correctly, the services have to know the context of the caller.
The most popular technic for identity propagation is extracting the context of the user within JSON Web Tokens.
The tokens are passed between the microservices and the API Gateway.
The propagated identity of the user can be extracted from the token, and the token's signature must be checked.
The microservices can perform authorization based on the identity of the client~\cite{barabanov2020authentication, dias2020microservices}.
The chosen authentication mechanism do have an impact on the identity propagation, this will be discussed in more detail in chapter~\ref{cha:authentication_mechanisms}.

\section{Authentication}



