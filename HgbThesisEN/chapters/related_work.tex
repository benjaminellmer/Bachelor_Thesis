\chapter{Related Work}
\label{cha:Related_Work}
This chapter summarizes the state-of-the-art concepts for microservice security and for public key based authentication.
Furthermore the technologies, which are used to implement the later discussed authentication mechanisms are described.
This chapter aims to give a good understanding of the whole domain and not only about service-to-service authentication.

\section{Microserivce Security}
Siriwardena and Dias~\cite{dias2020microservices} gave an exhensive guide of all topics related to microservice security. 
They separated the security of an microservice deployment into edge-level security and service-level security.
The microservice architecture is based on separating the system in multiple parts.
It is not sufficient to secure the system only on the edge-level or only on the service-level.
Each part of the system has to be secured properly regarding confidentiality, authentication and authorization.

\subsection{Edge-level security}
Edge-level security is defined as the security mechanisms that protect the resources within the deployment from attackers outside the deployment.
The API Gateway is responsible for the edge security it is the only entry point to the microservice deployment.
All requests targeted for the APIs of the services are intercepted by the API Gateway.
After validation of the requests, it dispatches the requests to the microservices.
The main tasks of an API Gateway are authentication of the end user, authorization, and throttling.
It authenticates the end user using access tokens, which come from access delegation technologies like OAuth 2.0 or Open ID Connect~\cite{siriwardena2014advanced}.
Due to outourcing the end user authentication to the API Gateway, the end user authentication has to be performed only once and not by every service.
The API Gateway is able to perform simple authorization assertions. 
As soon as the authorization gets more granular, the microservices have to perform it on their own, because the API Gateway does not know the business logic.

\subsection{Service-level security}
Service-level security is defined as the security mechanisms that protect the communication among the microservices.
According to Barabanov and Makrushin~\cite{barabanov2020authentication} service-level security can be decomposed into the sub functions service-level authentication, service-level authorization, and external identity propagation.
Service-level security can either be implemented by the microservices themselves or by a service-mesh.
A service mesh can be seen as a dedicated infrastructure layer, which manages the service-to-service communication of containerized services.
In a typical microservice deployment with a service mesh, each microservice has its service proxy, which works transparently~\cite{dias2020microservices}
The servce mesh is responsible for service discovery, routing, load balancing, traffic configuration, authentication authorization and monitoring~\cite{chandramouli2019microservices}.

\subsubsection{Service-to-service-authentication} 
\label{sec:service-to-service-authentication}
Authentication is the process of identifying the communication partner to protect a system from spoofing.
Since the communication among microservices is done using remote calls, their communication has to provide authentication.
Service-to-service authentication can be implemented in the following ways~\cite{dias2020microservices}:
\begin{itemize}
    \item Trust the Network (TTN)
    \item mutual Transport Layer Security (mTLS)
    \item self signed JSON Web Tokens (JWTs)
\end{itemize}
Trust the Network is a security approach, that is based on the assertion that nobody has access to the components within a network perimeter.
All components rely on the network security.
Nevertheless, internal misbehaviour can lead to exploits allowing attackers intrude into the network perimeter and exploit the microservices~\cite{zaheer2019eztrust}. 
Therefore the industry is heading towards zero-trust networks and the TTN approach became deprecated~\cite{dias2020microservices}.

Service-to-servie authentication based on mTLS and self signed JWTs will be discussed in more detail in chapter~\ref{cha:authentication_mechanisms}.

\subsubsection{Service-level authorization} 
\label{sec:service-level-authorization}
Authorization defines the tasks that a principal is allowed to perform on a system.
It requires that the principal is already authenticated because the authorization is performed based on the identity~\cite{siriwardena2014advanced}. 
Service-level authorization gives the microservices more control to enforce access control.
The authorization is usually performed using policy decision point (PDP) models like the centralized PDP model or the embedded PDP model~\cite{dias2020microservices, barabanov2020authentication}.
Proper service-to-service mechanisms are a precondition for service-to-service authorization, since the authorization can be bypassed with insufficient authentication~\cite{siriwardena2014advanced}.

\subsubsection{External entity identity propagation} 
\label{sec:external-entity-identity-propagation}
In order to perform the authorization correctly, the services have to know the context of the caller.
The most popular technic for identity propagation is extracting the context of the user within JSON Web Tokens.
The tokens are passed between the microservices and the API Gateway.
The propagated identity of the user can be extracted from the token, and the token's signature must be checked.
The microservices can perform authorization based on the identity of the client~\cite{barabanov2020authentication, dias2020microservices}.
The chosen authentication mechanism do have an impact on the identity propagation, this will be discussed in more detail in chapter~\ref{cha:authentication_mechanisms}.

\section{Public key-Based Authentication}
Authentication can be performed in many ways.
The authentication mechanisms which are discussed in chapter~\ref{cha:authentication_mechanisms} are both based on public key cryptography.
Public key cryptography provides higher security than symmetric cryptography.
Therefore it is the preferred method to implement authentication mechanisms, altough it requires higher computation and communication costs than symmetric cryptography~\cite{pubkeycrypto}.


\subsection{Public key cryptography}

\subsection{PKI}

\subsection{Key Management} \label{sec:key_management}

% \cite{canetti2016universally}

\section{Technologies}
\subsection{X509.Certificate}
X.509 certificates assure the users of a public key that the associated person or system owns the private key by binding public keys to subjects.
Certificate authorities sign certificates and each communication partner who trusts the CA trusts the certificates signed by it.
The most significant advantage of certificates is that they can be exchanged using untrusted communication channels because the signatures are not valid anymore when the contents of a certificate are changed.
Therefore manipulations can be detected, and manipulated certificates can be declined~\cite{x509rfc}.

\subsubsection{Trust Path}
When the client of a service wants to consume a service, which is hosted on a server, it has to obtain the server's certificate.
If the client does not know the public key of the CA who signed the server's certificate, he has to obtain it.
Obtaining the public key often results in chains because the client may have to work his way up until he reaches a CA he trusts.
Such chains are also called certification paths.
The way in which the clients can retrieve the CA certificates can be configured by the CA.

\subsubsection{Fields}
Depending on the version, a certificate can include more or less information.
The information is always stored inside the tbsCertificate, signatureAlgorithm, and signatureValue fields and can be expanded using extensions.

The TBSCertificate contains the data of the certificate, including the following information:
\begin{itemize}
    \item Subject of the certificate
    \item Issuer of the certificate
    \item public key of the subject
    \item Validity period
    \item Additional information
\end{itemize}

The signatureAlogrithm field stores the information, which cryptographic algorithm was used to sign the certificate.
Algorithms are declared by their identifier, the "OBJECT IDENTIFIER".
The most commonly used algorithms are the RSA\footnote{Rivest Shamir Adleman} algorithm and the Digital Signature Algorithm (DSA)~\cite{x509rfc}.

The signatureValue field contains the value of the digital signature.
It is obtained by signing the content of the tbsCertificate, using the algorithm specified in the signatureAlgorithm field.
The signature is used to verify the validity of the information embedded in the tbsCertificate field.

\subsection{JSON Web Token}
A JSON Web Token (JWT) is a container, which can carry authentication and authorization assertions and further information in a cryptographically safe manner.
An authentication assertion can be anything, which authenticates the user.
Usually, usernames or e-mail addresses are used to identify a user uniquely.
An authorization assertion can be any information about the access permissions of a user.
For example, a JWT can include the information, whether the user is an admin or an unprivileged user~\cite{dias2020microservices}. 

\subsubsection{Structure}
A JWT is decomposed into the header, the payload, and the signature.
The three parts are concatenated and separated by a dot~\cite{jwtdocauth0}.
A valid JWT could look like the JWT shown in figure~\ref{fig:myjwt}.
\begin{figure}
    \textcolor{red}{Header}.
	\textcolor{blue}{Payload}.
	\textcolor{darkgreen}{Signature} \\ \\
    \textcolor{red}{eyJhbGciOiJIUzI1NiIsInR5cCI6IkpXVCJ9}.
	\textcolor{blue}{eyJzdWIiOiIxMjM0NTY3ODkiLCJpYXQi\\OjE1MTYyMzkwMjIsInVzZXJuYW1lIjoiYmVuamFtaW4uZWxsbWVyIiwiZW1haWw\\iOiJiZW5qYW1pbi5lbGxtZXJAeWFob28uY29tIiwiYWRtaW4iOmZhbHNlfQ}.
	\textcolor{darkgreen}{0ksqN71oloNvq3IrY7w72uoTgPz9Gpn08p-KSbFulY0}
    \caption{Sample JSON Web Token}
    \label{fig:myjwt}
\end{figure}

The \textbf{header} contains the metadata related to the JWT, which is usually the type of the token and the signature algorithm.
The specification defines that only HS256\footnote{HMAC SHA-256} and none algorithm must be implemented  by conforming JWT implementation.
It is recommended to additionally implement the algorithms RS256 and ES256\footnote{Elliptic Curve Digital Signature Algorithm (ECDSA) with 256-bit key}~\cite{jwtdocauth0, jwtrfc}.
The base64 encoded header is the first part of the JWT.

The \textbf{payload} is a set of registered and custom claims.
A claim is a piece of information about an entity.
The JWT specification defines registered claims, which are not mandatory for all cases but should provide a good starting point for a set of useful claims to ensure interoperability.
Custom claims can be defined by the software architects, on their own, depending on their needs.
The custom claims registered in the IANA registry are called public claims, and those not registered in the IANA registry are called private claims~\cite{jwtdocauth0, jwtrfc}.
The base64 encoded payload is the second part of the JWT.

The chosen signature algorithm signs the base64 encoded header, the base64 encoded payload, and a secret.
The \textbf{signature} provides integrity for the message, and if it was signed with a private key, it additionally provides authentication~\cite{jwtdocauth0}.
The base64 encoded signature is the third part of the JWT.

\subsection{Transport Layer Security}
The Transport Layer Security (TLS) Protocol provides authentication, integrity, and confidentiality for the communication between two parties.
It consists of two layers, the handshake protocol, and the record protocol~\cite{turnertls}.

\subsubsection{mTLS} \label{sec:mtls}
TLS itself is also called one-way TLS because it helps the client to identify the server, but not the server to identify the client.
Therefore mutual TLS (mTLS) was introduced to provide authentication in both directions.
The client and the server must own a private/public key pair, so it is more suited for the communication between two systems and not between users and servers~\cite{dias2020microservices}. 

\subsubsection{Handshake Protocol}
The handshake protocol is responsible for negotiating a cipher suite and for the authentication using X.509 certificates.
The cipher suite declares the key exchange algorithm, the signature algorithm, the symmetric encryption algorithm, including the mode of the encryption algorithm and the hashing algorithm~\cite{turnertls, kurbatov2021design}.
The handshake varies on the key exchange method, but it can be separated into the following steps~\cite{krawczyk2013security}:
\begin{enumerate}
    \item The server and the client exchange Hello messages
    \item The server sends its certificate to the client
    \item The client sends a pre-master secret to the server and if mTLS is used, the client sends his certificate to the server
    \item The client and the server finish the handshake, using the independently computed master secret
\end{enumerate}
The steps of the handshake will be explained in more detail in chapter~\ref{sec:tlshandshake_details}.

\subsubsection{Record Protocol}
The record protocol provides a secure channel for the communication between the parties.
This is done by using the algorithms declared in the cipher suite.
Confidentiality is assured, using symmetric encryption, and integrity is provided by Message Authentication Codes (MAC)~\cite{kurbatov2021design, krawczyk2013security}.

